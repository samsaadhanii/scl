\vspace*{15pt}
Ternary trees are more complex than tries, but use slightly less storage.
Access is potentially faster in balanced trees than tries. 
A good methodology seems to use tries for edition, and to translate them
to balanced ternary trees for production use with a fixed lexicon.

The ternary version of our English lexicon takes 3.6Mb, a savings of 20\%
over its trie version using 4.5Mb. After dag minimization, it takes 1Mb,
a savings of 10\% over the trie dag version using 1.1Mb. 
In the case of our Sanskrit lexicon index, the trie takes 221Kb and the tertree
180Kb, whereas shared as dags the trie takes 103Kb and the tertree 96Kb.

\section{Decorated Tries for Inflected Forms Storage}

\subsection{Decorated Tries}

A set of elements of some type $\tau$ may be identified as its
characteristic predicate in $\tau\rightarrow bool$. A trie with boolean 
information may similarly be generalized to a structure
representing a map, or function from words to some target type, by storing
elements of that type in the information slot. 

In order to distinguish absence of information, we could use a type
{\sl (option info)} with constructor {\sl None}, presence of value
{\sl v} being indicated by {\sl Some(v)}. We rather 
choose here a variant with lists,
which are versatile to represent sets, feature structures, etc. Now we
may associate to a word a non-empty list of information of polymorphic
type $\alpha$, absence of information being encoded by the empty list.
We shall call such associations a {\sl decorated trie}, or 
{\sl deco} in short.
