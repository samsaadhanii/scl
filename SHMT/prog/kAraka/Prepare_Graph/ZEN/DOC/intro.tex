%\documentclass[11pt]{article}
%\pagestyle{plain}

\def\lbr{\langle} %<
\def\rbr{\rangle} %>
\def\lsq{[} %[
\def\rsq{]} %]
\def\R{{\cal R}} % script R
\def\otrema{\"o} % otrema necessary for balance of "..." in comments.
\def\skip{\vspace{10pt}}

%\def\url{}

%\begin{document}
\begin{center}
\vspace*{24pt}
{\Large The Zen Computational Linguistics Toolkit}\\[10pt]
{\Large Version 3.2}\\[15pt]
{November 24th, 2013}\\[15pt]
{\large G\'erard Huet}\\[10pt]
{\large Copyright \copyright ~2002-2013 INRIA}\\[20pt]
\end{center}

\tableofcontents

\begin{abstract}
We present in this document a few fundamental structures useful for
computational linguistics. 

The central structure is that of lexical tree, or {\sl trie}.
A crucial observation is that a trie is isomorphic to the state space
of a deterministic acyclic automaton. More complex finite-state
automata and transducers, deterministic or not, and cyclic or
not, may be represented as tries decorated by extra information. Thus we
obtain a family of structures underlying lexicon-directed linguistic
processes.

First we describe plain tries, which are adequate to represent lexicon indexes.
Then we describe decorated tries, or {\sl decos}, which are appropriate to
represent symbol tables, and dictionaries associating with the lexicon
grammatical or other informations. We then describe how to represent
maps and more generally invertible relations between lexicons. We call these
structures lexical maps or {\sl lexmaps}. Lexmaps are appropriate for instance
to associate inflected forms to lexicon stems and roots, using morphological
operations. Such lexmaps are invertible in the sense that we may retrieve
from the lexmap entry of a inflected form the stems and operations from which
it may be obtained. Finally we show how lexicon directed transducers may
be represented using tries decorated with choice points. Such transducers
are useful to describe segmentation and taggings processes.

All data structures and algorithms are described
in a computational metalanguage called \verb:Pidgin ML:. \verb:Pidgin ML: is a 
publication language for the ML family of programming languages. All the
algorithms described here could be described as well in Standard ML
or in Objective CAML, to cite two popular ML implementations, or in the
lazy functional language Haskell. They could also be described in a 
programming language such as LISP or Scheme, but the strong typing discipline
of ML, supporting polymorphism and modules, is an insurance that computations
cannot corrupt data structures and lead to run-type errors. 
An initial chapter of these notes gives a quick overview of \verb:Pidgin ML:. 

The resulting design may be considered as the reference implementation of a Free 
Computational Linguistics Toolkit. It may turn useful as an ``off the shelf''
toolkit for simple operations on linguistics material. Due to its
lightweight approach we shall talk of the Zen CL Toolkit. 

This toolkit was abstracted from the Sanskrit ML Library, which constitutes
its first large-scale application. Thus some of this material already
appeared in the documentation of the Sanskrit Segmenter algorithm,
which solves Sandhi Analysis \cite{2004-Huet-1}.
%The Sanskrit Library Documentation, a companion to this document,
%is available at \url{http://pauillac.inria.fr/~huet/SKT/DOC/doc.ps} under
%format postscript, \verb|doc.pdf| under format pdf, 
%and \verb|doc.html| under format html.

This document was automatically generated from the code of the toolkit
using the Ocamlweb package of Jean-Christophe Filli\^atre,
with the Latex package, in the literate programming style pioneered by
Don Knuth. 
%The Html version uses the Hevea Tex-to-Html translator of Luc Maranget.

\end{abstract}

\part{Dictionaries}

\section{Pidgin ML}

We shall use as {\sl meta language} for the description of our algorithms
a pidgin version of the functional language ML 
\cite{ML-LCF,MLer,paulson,caml}. 
Readers familiar with ML
may skip this section, which gives a crash
overview of its syntax and semantics.

